\documentclass[t,11pt,british,english, top=1.0in]{beamer}
%\usetheme{amcg}
%\usetheme{iclpt}
\usepackage{url}
\usepackage{hyperref}
\usepackage{graphics}
\usepackage{amsmath}
\usepackage{amssymb}
\usepackage{tikz}
\usetikzlibrary{calc,decorations.pathreplacing}
\usepackage{natbib}
        %You can use the package \textbf{pgfpages} 
        %to arrange your slides for printing. This is also explained
        %in the \textbf{beamer} documentation.

\newcommand{\tensor}[1]{\overline{\overline{#1}}}
\newcommand{\tautens}{\tensor{\tau}}

\setbeamerfont{framesubtitle}{size=\normalsize}
\setbeamerfont{framesubtitle}{size=\normalsize}
%\setbeamertemplate{frametitle}[default][center]
%\setbeamersize{text margin left=6mm}

\usetheme{Madrid}
\usecolortheme{orchid}

%gets rid of bottom navigation bars
\setbeamertemplate{footline}[page number]{}
\setbeamertemplate{headline}{}

%gets rid of navigation symbols
\setbeamertemplate{navigation symbols}{}

\begin{document}
\title{Introduction to programming for Geoscientists\\\vspace*{5mm}Revision Lecture 2}
\author{} 
\date{} 

\frame{\titlepage} 

\frame{
   \frametitle{Classes and Objects}
   \framesubtitle{Definitions} 
   \begin{itemize}
    \item {\color{red}Class}: a programming construct that allows a set of objects with common properties to be described as a single package that encapsulates data (attributes) and functions (behaviour).\vspace*{2mm}
    \begin{itemize}
     \item They make programs more managable.
     \item They allow information hiding.
    \end{itemize}\vspace*{2mm}    
    \item {\color{red}Object}: a specific {\color{red}instance} of a class.\vspace*{2mm}
    \item {\color{red}Instantiation}: the process of creating an object from a class.\vspace*{15mm}
    
    \item A {\color{red}class} is like a {\color{red}blueprint/template} from which {\color{red}objects} can be {\color{red}created/instantiated}.\vspace*{2mm}
   \end{itemize}
}

\frame{
   \frametitle{Classes and Objects}
   \framesubtitle{Motivation} 
    \begin{minipage}[t]{\textwidth}
       \texttt{def print\_data(id, name, course):}
    
       \ \ \ \ \ \indent\texttt{print "Student id: \%d, name: \%s, course: \%s" \% \\(id, name, course)}
       
       \texttt{ }
       
       \texttt{student1\_id = 1}
       
       \texttt{student1\_name = "Bob"}
       
       \texttt{student1\_course = "Geology"}
       
       \texttt{student2\_id = 2}
       
       \texttt{student2\_name = "Alice"}
       
       \texttt{student2\_course = "Computer science"}
       
       \texttt{print\_data(student2\_id, student2\_name, student2\_course)}
    \end{minipage}\vspace*{2mm}
}

\frame{
   \frametitle{Classes and Objects}
   \framesubtitle{Motivation} 
    \begin{minipage}[t]{\textwidth}
       \texttt{class Student:}
       
       \ \ \ \ \ \indent\texttt{def \_\_init\_\_(self, id, name, course):}
          
       \ \ \ \ \ \ \ \ \ \ \indent\texttt{self.id = id}
       
       \ \ \ \ \ \ \ \ \ \ \indent\texttt{self.name = name}
 
       \ \ \ \ \ \ \ \ \ \ \indent\texttt{self.course = course}
       
       \ \ \ \ \ \indent\texttt{def print\_data(self):}
       
       \ \ \ \ \ \ \ \ \ \ \indent\texttt{print "Student id: \%d, name: \%s, course: \%s" \% \\(self.id, self.name, self.course)}
       
       \texttt{ }
              
       \texttt{student1 = Student(1, "Bob", "Geology")}
       
       \texttt{student2 = Student(2, "Alice", "Computer science")}
       
       \texttt{student2.print\_data()}
    \end{minipage}\vspace*{2mm}
}

\frame{
   \frametitle{Classes and Objects}
   \framesubtitle{Examples} 
   \begin{itemize}   
    \item A cake ({\color{red}object}) is baked in the oven ({\color{red}instantiated}) once prepared from a generic cake recipe ({\color{red}class}):\vspace*{3mm}
    \begin{itemize}
      \item data/attributes: flavour, number of slices
      \item functions/behaviour: remove slice, expire
    \end{itemize}\vspace*{5mm}
    \item A human ({\color{red}object}) is born ({\color{red}instantiated}) with physical characteristics defined by a genetic makeup ({\color{red}class}):\vspace*{3mm}
    \begin{itemize}
      \item data/attributes: eye colour, hair colour
      \item functions/behaviour: sleep, drink, eat
    \end{itemize}\vspace*{2mm}
   \end{itemize}
}

\frame{
   \frametitle{Classes and Objects}
   \framesubtitle{Terminology} 
   \begin{itemize}
    \item Variables belonging to a class are known as {\color{red}attributes}.\vspace*{2mm}
    \item Functions belonging to a class are known as {\color{red}methods}.\vspace*{2mm}
    \item It is good practice to change attributes via {\color{red}get/set} methods, not directly.
   \end{itemize}
}

\frame{
   \frametitle{Classes and Objects}
   \framesubtitle{Example} 
   \begin{itemize}
    \item General example of a class definition in Python:\vspace*{2mm}
    \begin{minipage}[t]{\textwidth}
       \texttt{class ClassName:}
    
       \ \ \ \ \ \indent\texttt{def \_\_init\_\_(self, input1, input2):}
       
       \ \ \ \ \ \ \ \ \ \ \indent\texttt{self.a = input1}
       
       \ \ \ \ \ \ \ \ \ \ \indent\texttt{self.b = input2}
       
       \ \ \ \ \ \indent\texttt{def method1(self, input1):}
       
       \ \ \ \ \ \ \ \ \ \ \indent\texttt{print "Hello \%s!" \% input1}
       
       \ \ \ \ \ \indent\texttt{def method2(self):}
       
       \ \ \ \ \ \ \ \ \ \ \indent\texttt{print "a = \%d, b = \%d" \% (self.a, self.b)}
                  
       \texttt{A = ClassName(5, 10)}

       \texttt{A.method1("world")} $\rightarrow$ ``Hello world!''
       
       \texttt{A.method2()} $\rightarrow$ ``a = 5, b = 10''
    \end{minipage}\vspace*{2mm}
    \item \texttt{self} can be thought of as the object that is calling one of the methods.
    \item \texttt{\_\_init\_\_} is a special method used to initialise/setup objects.
   \end{itemize}
}

\frame{
   \frametitle{Classes and Objects}
   \framesubtitle{Cake} 
   \begin{itemize}
    \item The following example describes cake:
    \begin{minipage}[t]{\textwidth}
       \texttt{class Cake:}
    
       \ \ \ \ \ \indent\texttt{def \_\_init\_\_(self, cake\_type):}
       
       \ \ \ \ \ \ \ \ \ \ \indent\texttt{self.type = cake\_type}
       
       \ \ \ \ \ \ \ \ \ \ \indent\texttt{self.number\_of\_slices = 10}
       
       \ \ \ \ \ \indent\texttt{def eat\_slice(self):}
              
       \ \ \ \ \ \ \ \ \ \ \indent\texttt{self.number\_of\_slices = self.number\_of\_slices-1}
       
       \ \ \ \ \ \ \ \ \ \ \indent\texttt{print "\%d slices remaining." \% self.number\_of\_slices}     
       
                  \ \ \ \ \ \ \ \ \ \
                  
       \texttt{A = Cake("Chocolate")}
       
       \texttt{B = Cake("Lemon")}
           
           \ \ \ \ \ \ \ \ \ \
           
       \indent\texttt{print A.type} $\rightarrow$ ``Chocolate''
       
       \indent\texttt{print B.type} $\rightarrow$ ``Lemon''
       
       \indent\texttt{B.eat\_slice()} $\rightarrow$ ``9 slices remaining.''
    \end{minipage}\vspace*{2mm}
    
   \end{itemize}
}

\frame{
   \frametitle{NumPy Arrays}
   \framesubtitle{Definition} 
   \begin{itemize}
    \item {\color{red}Arrays}: data structures which comprise a finite number of {\color{red}elements}/items/values.\vspace*{2mm}
    \item Similar to lists (or lists of lists), but are of a {\color{red}fixed size} and can only contain {\color{red}one data type}.\vspace*{2mm}
    \item Arrays are generally {\color{red}faster than lists} because array elements are stored in {\color{red}contiguous} areas of memory.
   \end{itemize}
}

\frame{
   \frametitle{NumPy Arrays}
   \framesubtitle{linspace and zeros} 
   \begin{itemize}
    \item Two useful functions for creating arrays:\vspace*{2mm}
      \begin{itemize}
      \item {\color{red}linspace(start, end, n)}: creates an array of $n$ uniformly distributed points in the interval [start, end].\vspace*{2mm}
      \item {\color{red}zeros(n)}: creates an array of $n$ elements that are all initialised to zero.
     \end{itemize}\vspace*{2mm}
    \item Or...simply define as a list of lists (in 2D) and cast/convert to an array:\vspace*{2mm}
    \begin{minipage}[t]{0.4\textwidth}
    $$
\left\lbrack\begin{array}{ccc}
0 & 12 & -1\cr
-1 & -1 & -1\cr
11 & 5 & 5 
\end{array}\right\rbrack
$$
    \end{minipage}    \vspace*{2mm}
    \begin{minipage}[t]{0.4\textwidth}
       \texttt{a = [ [0, 12, -1],}
    
       \texttt{[-1, -1, -1],}
       
       \texttt{[11, 5, 5] ]}
       
       \texttt{a = array(a)}
       
    \end{minipage}\vspace*{2mm}

    \item Remember to include \texttt{from numpy import *} in your program.
   \end{itemize}
}

\frame{
   \frametitle{NumPy Arrays}
   \framesubtitle{Referencing/accessing elements} 
   \begin{itemize}
    \item Referencing/accessing elements in an array is the same as referencing list elements.\vspace*{2mm}
    \item \texttt{a[i][j]} accesses the element at row \texttt{i} and column \texttt{j}.\vspace*{2mm}
    \item {\color{red}R}ow fi{\color{red}R}st, {\color{red}C}olumn se{\color{red}C}ond.\vspace*{2mm}
   \end{itemize}
}

\frame{
   \frametitle{NumPy Arrays}
   \framesubtitle{Vectorised functions} 
   \begin{itemize}
    \item A {\color{red}vectorised} function can accept an array as its input...\vspace*{2mm}
    \item ...and for each element of that array, compute the result...\vspace*{2mm}
    \item ...and output all results in a new array.
    \begin{minipage}[t]{\textwidth}
       \texttt{from numpy import *}
       
       \texttt{a = linspace(0, 1, 10)}
    
       \texttt{result = sin(a) \# Result is an array here.}
    \end{minipage}\vspace*{2mm}
    \item This is like doing:
    \begin{minipage}[t]{\textwidth}
       \texttt{from numpy import *}
       
       \texttt{a = linspace(0, 1, 10)}
       
       \texttt{result = zeros(10)}
    
       \texttt{for i in range(0, 10):}
       
       \ \ \ \ \ \indent\texttt{result[i] = sin(a[i])}
    \end{minipage}\vspace*{2mm}
    \item But with vectorised functions, this \texttt{for} loop is implicit.
   \end{itemize}
}

\frame{
   \frametitle{Strings}
   \framesubtitle{Definition} 
   \begin{itemize}
    \item {\color{red}String}: a {\color{red}sequence of characters}, terminated by an {\color{red}end-of-line marker}.\vspace*{2mm}
    \item Each character in a string can be accessed in the same way as elements of a list or array:
    \begin{minipage}[t]{\textwidth}
       \texttt{s = "hello"}
       
       \texttt{print s[0]} $\rightarrow$ ``h''
    
       \texttt{print s[2]} $\rightarrow$ ``l''
    \end{minipage}\vspace*{2mm}
    \item \texttt{split} breaks up strings wherever a user-defined {\color{red}delimiter} is encountered.\vspace*{2mm}
    \item For example, if the delimiter is a comma:
    \begin{minipage}[t]{\textwidth}
       \texttt{s = "hello world, Python is really awesome."}
       
       \texttt{print s.split(",")} $\rightarrow$ [``hello world'', `` Python is really awesome'']
    \end{minipage}\vspace*{2mm}
    \item Remember: Strings are {\color{red}immutable}/{\color{red}constant} data structures. They cannot be modified once defined.
   \end{itemize}
}

\frame{
   \frametitle{Files}
   \framesubtitle{Reading} 
   \begin{itemize}
    \item Open a file (for reading) using \texttt{f = open("file\_name\_here.txt", "r")}.\vspace*{2mm}
    \item A file can be thought of as a {\color{red}list of strings}, with each string being a single line of the file.\vspace*{2mm}
    \item We can select one line at a time using \texttt{f.readline()}, ...\vspace*{2mm}
    \item ...or select all the lines in the file using \texttt{f.readlines()}.\vspace*{2mm}
    \item It is good practice to {\color{red}close} the file (once it is no longer needed) with \texttt{f.close()}
   \end{itemize}
}

\frame{
   \frametitle{Files}
   \framesubtitle{Writing} 
   \begin{itemize}
    \item Open a file (for writing) using \texttt{f = open("file\_name\_here.txt", "w")}.\vspace*{2mm}
    \item Write a string to the file using \texttt{f.write(string\_to\_write\_here)}.\vspace*{2mm}
    \item Once again, remember to close the file after use.
   \end{itemize}
}

\frame{
   \frametitle{Dictionaries}
   \framesubtitle{Definition} 
   \begin{itemize}
    \item {\color{red}Dictionary}: a data structure whose elements are {\color{red}key-value pairs}.\vspace*{2mm}
    \item The key does not have to be an integer.\vspace*{2mm}
    \item Example: \texttt{d = \{"Barcelona":11.0, "Lleida":6.0, "Tarragona":8.0 \}}\vspace*{2mm}
    \item \texttt{d.keys()} $\rightarrow$ [``Barcelona'', ``Lleida'', ``Tarragona'']\vspace*{2mm}
    \item \texttt{d.values()} $\rightarrow$ [11.0, 6.0, 8.0]\vspace*{2mm}
    \item Items can be added using \texttt{b[new\_key\_here] = value\_here}.\vspace*{2mm}
    \item ...or existing items can be accessed using the key: \texttt{print b["Barcelona"]}.\vspace*{2mm}
   \end{itemize}
}

\end{document}
